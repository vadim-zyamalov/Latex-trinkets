\documentclass[12pt, rusmathsym]{nirreport}

\usepackage{cmap}
\usepackage[utf8]{inputenc}
\usepackage[T2A]{fontenc}
\usepackage[russian]{babel}

\usepackage{graphicx}
\usepackage{float}

\setlength\parindent{15mm}

\usepackage{lipsum}

\NirOrgLongName{Черноморское отделение Арбатовской конторы\\по заготовке рогов и копыт}%
%\NirOrgShortName{РАНХиГС}%
\NirTitle{}
\NirSubject{Удар автопробегом по разгильдяйству}
\NirFinal
\NirApprover{Великий комбинатор}{О.~Б.~М.~Бендер-бей}%
\NirSupervisor{Нарушитель\\конвенции}{М.~С.~Паниковский}
\NirTown{Черноморск}
\NirYear{1930}

% \def\abstractname{Реферат}

% \addto\captionsrussian{%
%   \def\contentsname{Содержание}%
%  \def\bibname{Список~использованных~источников}%
% }


%\addbibresource{bibliography.bib}

% https://tex.stackexchange.com/a/55706
\usepackage[backend=biber, style=gost-numeric, uniquelist=false, movenames=false]{biblatex}

\addbibresource{bibliography.bib}

\begin{document}

\titlepage

\begin{abstract}
  Привет, это пример!

  Кое-чего не хватает...
\end{abstract}

\tableofcontents

\chapter{Глава}

\begin{itemize}
\item Раз
\item Два
\end{itemize}

\begin{enumerate}
\item Раз
\item Два
\end{enumerate}

\lipsum[1]

\section{Раздел}

\lipsum[1]

\subsection{Глубже}

\lipsum[1]

\subsubsection{Подраздел}

\lipsum[1]
\begin{equation}
  E \geq mc^2
\end{equation}
\begin{equation}
  E \leq mc^2
\end{equation}
\lipsum[1]

\subsubsection{Подподраздел}

\lipsum[1]

\begin{table}[ht]
  \caption{Как интересно то...}
  \centering
  \begin{tabular}{ccc} \hline
    A           & B & C            \\ \hline
    1           & 2 & 3            \\
    \(e^{\pi}\) & 0 & \(\Uparrow\) \\ \hline
  \end{tabular}
\end{table}

\chapter{Еще одна глава}

\lipsum[1]

\begin{table}[!ht]
  \caption{Как интересно то...}
  \centering
  \begin{tabular}{ccc} \hline
    A           & B & C            \\ \hline
    1           & 2 & 3            \\
    \(e^{\pi}\) & 0 & \(\Uparrow\) \\ \hline
  \end{tabular}
\end{table}

\lipsum[1]

\begin{figure}[!ht]
  \begin{center}
    \includegraphics[height=.1\textheight]{img/leonardo.png}
  \end{center}
  \caption{Тест}
\end{figure}

\chapter*{Заключение}

Ссылка~\cite{Pup09}

\lipsum[1]

% \bibliographystyle{ugost2008}
% \bibliography{bibliography}
\printbibliography

\appendix

\chapter{Новое приложение}

\[ E = m c^2 \]

\chapter{И еще одно}

\[ F = m a \]

\end{document}

%!TEX root = ../main.tex

%%%%%%%%%% Програмный код %%%%%%%%%%
\usepackage{minted}
% Включает подсветку команд в программах!
% Нужно, чтобы на компе стоял питон, надо поставить пакет Pygments, в котором он сделан, через pip.

% Для Windows: Жмём win+r, вводим cmd, жмём enter. Открывается консоль.
% Прописываем pip install Pygments
% Заходим в настройки texmaker и там прописываем в PdfLatex или XelaTeX:
% pdflatex -shell-escape -synctex=1 -interaction=nonstopmode %.tex

% Для Linux: Открываем консоль. Убеждаемся, что у вас установлен pip командой pip --version
% Если он не установлен, ставим его: sudo apt-get install python-pip
% Ставим пакет sudo pip install Pygments

% Для Mac: Всё то же самое, что на Linux, но через brew.

% После всего этого вы должны почувствовать себя тру-программистами!
% Документация по пакету хорошая. Сам читал, погуглите!

%%%%%%%%%% Математика %%%%%%%%%%
\usepackage{amsmath,amsfonts,amssymb,amsthm,mathtools}
%\mathtoolsset{showonlyrefs=true} % Показывать номера только у тех формул, на которые есть \eqref{} в тексте.
%\usepackage{leqno} % Нумерация формул слева

%%%%%%%%%% Шрифты %%%%%%%%%%
\usepackage{cmap}
\usepackage[utf8]{inputenc}            % выбор utf8 кодировки
\usepackage[T2A,T1]{fontenc}           % ещё немного кодировки
\usepackage[english, russian]{babel}   % выбор языка для документа
\usepackage{csquotes}
%\usepackage{fontspec}                  % пакет для подгрузки шрифтов
%\setmainfont{Times New Roman}          % задаёт основной шрифт документа
%\usepackage{unicode-math}              % пакет для установки математического шрифта
%\setmathfont{Asana-Math.otf}           % шрифт для математики
%\setmonofont{Courier New}
%\setmathfont[range=\int]{Neo Euler}   % Конкретный символ из конкретного шрифта

%%%%%%%%%% Другие приятные пакеты %%%%%%%%%%

\usepackage{verbatim}       % для многострочных комментариев
\usepackage{enumitem}       % дополнительные плюшки для списков
%  например \begin{enumerate}[resume] позволяет продолжить нумерацию в новом списке

\usepackage{todonotes}      % для вставки в документ заметок о том, что  осталось сделать
% \todo{Здесь надо коэффициенты исправить}
% \missingfigure{Здесь будет Последний день Помпеи}
% \listoftodos --- печатает все поставленные \todo'шки

%%%%%%%%%% Гиперссылки %%%%%%%%%%
\usepackage{xcolor}            % разные цвета
\usepackage{hyperref}
\hypersetup{
	unicode=true,           % позволяет использовать юникодные символы
	colorlinks=true,       	% true - цветные ссылки, false - ссылки в рамках
	urlcolor =blue,         % цвет ссылки на url
	linkcolor=black,        % внутренние ссылки
	citecolor=black,        % на библиографию
	breaklinks              % если ссылка не умещается в одну строку, разбивать ли ее на две части?
}
